\documentclass[12pt,letterpaper]{article}

\def \mylastname {Marting}
\def \myname {Matthew \mylastname{}}

\usepackage[margin=1in]{geometry}

\linespread{2}

\usepackage{times}

\usepackage{fancyhdr}
\pagestyle{fancy}
\fancyhf{}
\rhead{\mylastname{} \thepage{}}
\renewcommand{\headrulewidth}{0pt}
\setlength{\headsep}{24pt}

\setlength{\parskip}{0pt}

\usepackage{titlesec}
\titleformat{\section}{\scshape{}}{}{}{}
\titlespacing*{\section}{0pt}{\baselineskip}{0pt}

\usepackage{calc}
\titleformat{\subsection}[runin]{\itshape{}}{}{}{}[\normalfont{}.]
\titlespacing{\subsection}{0pt}{0pt}{\widthof{. }-\widthof{.}}

\newenvironment{workscited}{
  \newcommand{\bibentry}{\noindent{}\hangindent=0.5in}
  \newpage{}
  {\centering{}Works Cited\par{}}
}{\newpage{}}

\begin{document}
\begin{flushleft}
  \myname{}\\
  {\centering{}Dependency Tree Linearization Literature Review\par{}}
  \setlength{\parindent}{0.5in}
  \section*{``The First Surface Realisation Shared Task:\\
  Overview and Evaluation Results"}
  The purpose of the Surface Realisation (SR) Task was to compare SR systems. An SR system produces a sentence, or realization, given as an input a representation thereof. At least here, all the representations comprised nodes and edges. Nodes corresponded to lexical units. The edge of one node to another described the nature of the first node's dependence on the second.

  To compare SR systems, the team organizing the SR Task established two common input formats: shallow and deep. As opposed to prior tests of SR systems, the SR Task's common input formats ensured that, within each kind of input representation (shallow or deep), each system received the same amount of information. That is, each \textit{shallow} SR system would receive the same amount of information as all the other \textit{shallow} SR systems; all the \textit{deep}, the same as all the other \textit{deep}. The team ``assessed three criteria in the human evaluations: Clarity, Readability and Meaning Similarity" (Belz et al., 221), and they also scored SR systems according to ``the following well-known automatic evaluation metrics: [. . .] \textsc{bleu}, [. . .] \textsc{nist}, [. . .] \textsc{meteor}, [. . .] \textsc{ter}" (Belz et al., 220). Differences in systems' realization qualities would depend only on the systems themselves.

  As the SR Task parsed randomly-selected sentences within a corpus, it retained their original forms and contexts. A sentence's original form would serve as a kind of ``human topline" (Belz et al., 217). Perhaps we could use such a topline for training?

  The shallow input format was ``a more `surfacey', syntactic representation of the sentence" (Belz et al., 218), while the deep format was ``closer to a semantic, more abstract, representation" (Belz et al., 218). As mentioned above, both representations comprised nodes and edges, where nodes corresponded to lexical units. As such, each node comprised the lemma; also present for both formats were, if applicable, the tense, number, ``tense and participle features and a sense tag id (as a suffix to the lemma)" (Belz et al., 218). % to-do: what are those?
  The shallow format's nodes, unlike those of the deep format, contain a part-of-speech (POS) tag. Together with the edges they comprise a dependency tree. Shallow edges have labels that signify the grammatical role of the node they point to to the node they originate from, such as \texttt{OBJ} (for object of a verb) or \texttt{SBJ} (for subject) (Natural Language Technology Group, 3). These syntactic edges may have more than one label. The strict arrangement of nodes and edges into a tree poses an issue for sentences such as ``He commissions and splendidly interprets fearsome contemporary scores" (Natural Language Technology Group, 3). ``scores" is the object of both ``commissions" and ``interprets", but since only one edge may point to each node, it is only the object of ``commissions" (Natural Language Technology Group, 4). This is a definite disadvantage of using a tree structure, but one could parse to a tree and then perform ``a post-processing step which adds some of these missing relations, producing graph representations with multiple headed dependencies" (Natural Language Technology Group, 3).
  \subsection*{Potential Implementation} One could make a \texttt{Node} class and an \texttt{Edge} class. \texttt{Node} would have a \texttt{std::wstring lemma\_}. One could store the POS as either a \texttt{enum class} (given Apertium's recent adoption of the not-so-recent C++11 standard) or a \texttt{std::wstring}. If the set of course tags remains constant enough to hard-code, the first option would be better by far, since one could switch on the POS. Such a switch could compile to a couple comparisons to check the range and then a jump table, since letting the compiler determine the enumerators' values would make them contiguous; therefore, there would be no need to check for invalid intermediate values in the switch. This would pose an issue were it necessary to store POS in a file since a change in the set of enumerators might mean a change in their values, which would corrupt all existing files unless one stored POS in a different way, such as a string representation. One could also set the enumerators' values to be contiguous manually without the fear of such change. Tense and number, among other attributes, would need to be \texttt{Optional}. I cannot see how it would be more efficient to write a custom version for \texttt{Node}, since, while one could base whether a field exists off the POS, to do so would require calling a large function with a switch statement, which would be fast but nonetheless slower than a simple comparison with \texttt{nullptr}; and one would still need to allocate memory, which could be done rather easily in a constructor that would call the POS function once to determine which \texttt{Optional} fields to initialize with and without parameters. Making a separate class for each POS is out of the question, since there will be so many POS tags and they could change quickly. Quickly enough, and the second option, storing the POS in a \texttt{std::wstring}, would be better, since one could read in a new set of tags instead of recompiling. To be useful, the format must include what tags would have which attributes; otherwise, this too would have to be hard-coded, but in a less-efficient if-else chain of string comparisons.
  \begin{workscited}
    \bibentry{}Belz, Anja, et al. ``The First Surface Realisation Shared Task: Overview and Evaluation Results." \textit{Proceedings of the 13th European Workshop on Natural Language Generation (ENLG)}, Sept. 2011, Nancy, France, Association for Computational Linguistics, 2011, pp. 217--226, www.aclweb.org/anthology/W11-2832. Accessed 29 Nov. 2016.

    \bibentry{}``Generation Challenges 2011 Surface Realisation Shared Task: Documentation and Instructions for Participants." \textit{Natural Language Technology Group}, University of Brighton, 19 Apr. 2011, www.itri.brighton.ac.uk/home/Anja.Belz/pdf/SR-Task-2011-Doc.pdf. Accessed 29 Nov. 2016.
  \end{workscited}
\end{flushleft}
\end{document}
